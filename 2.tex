\documentclass[pmath451]{subfiles}

%% ========================================================
%% document

\begin{document}

    \section{Measurable Functions}
    
    \subsection{Measurable Functions}
    
    Let $\left( X,\mA \right), \left( Y,\mB \right)$ be measurable spaces. We care about functions $f:X\to Y$ which relay information about the measurable spaces.

    \begin{definition}{\textbf{Measurable} Function}
        Let $\left( X,\mA \right), \left( Y,\mB \right)$ be measurable spaces. We say $f:X\to Y$ is \emph{measurable} if
        \begin{equation*}
            f^{-1}\left( B \right) = \left\lbrace x\in X: f\left( x \right)\in B \right\rbrace \in\mA,\hspace{1cm}\forall B\in\mB.
        \end{equation*}
    \end{definition}
    
    \np Before we proceed, here is a convention that we are going to use. Let $\F$ be $\R$ or $\CC$ and let $\left( X,\mA \right)$. We say 
    \begin{equation*}
        \text{$f:X\to Y$ is measurable} \iff \text{$f$ is measurable with respect to $\left( X,\mA \right),\left( \F,\Bor\left( \F \right) \right)$}.
    \end{equation*}
    By Assignment 1, we see that
    \begin{equation*}
        \text{$f:X\to Y$ is measurable} \iff \text{for all open $B$, $f^{-1}\left( B \right)\in\mA$},
    \end{equation*}
    since $\Bor\left( \F \right)$ is generated by open subsets of $\F$. In case $\F=\R$, we can replace $B$ with open interval, since every open subset of $\R$ is a countable union of open intervals.

    \np Recall the following trick for analysis. Let $a<b$ in $\R$. Then
    \begin{equation*}
        \begin{aligned}
            \left( a,b \right] & = \bigcap^{\infty}_{n=1} \left( a,b+\frac{1}{n} \right) \\
            \left( a,b \right) & = \bigcup^{\infty}_{n=1} \left( a,b-\frac{1}{n} \right] \\
            \left[ a,b \right] & = \bigcap^{\infty}_{n=1} \left( a-\frac{1}{n},b+\frac{1}{n} \right) \\
            \left( a,\infty \right) & = \bigcup^{\infty}_{n=1} \left( a,a+n \right) \\
            \left( a,b \right] & = \left( -\infty,b \right] \cap \left( a,\infty \right) \\
                               & \vdots
        \end{aligned} .
    \end{equation*}
    That is, all interval types independently generate $\Bor\left( \R \right)$.

    \begin{prop}{}
        Let $\left( X,\mA \right)$ be a measurable space and let $f:X\to\R$. The following are equivalent.
        \begin{enumerate}
            \item $f$ is measurable.
            \item For all $\alpha\in\R$, $f^{-1}\left( \left( \alpha,\infty \right) \right)\in\mA$.
            \item For all $\alpha\in\R$, $f^{-1}\left( \left[ \alpha,\infty \right) \right)\in\mA$.
            \item For all $\alpha\in\R$, $f^{-1}\left( \left( -\infty,\alpha \right) \right)\in\mA$.
            \item For all $\alpha\in\R$, $f^{-1}\left( \left( -\infty,\alpha \right] \right)\in\mA$.
        \end{enumerate}
    \end{prop}

    \rruleline

    \clearpage

    \begin{prop}{}
        Let $\left( X,\mA \right)$ be a measurable space and let $f:X\to\CC$. The following are equivalent. Then
        \begin{equation*}
            \text{$f$ is measurable} \iff \text{$\re\circ f$ and $\im\circ f$ are measurable}.
        \end{equation*}
    \end{prop}

    \begin{proof}[Proof Sketch]
        ($\impliedby$) Every open $U\subseteq\CC$ can be written as a countable union of open rectangles $\left( a,b \right)\times\left( c,d \right)$. Then
        \begin{equation*}
            f^{-1}\left( \left( a,b \right)\times\left( c,d \right) \right) = \left( \re\circ f \right)^{-1}\left( \left( a,b \right) \right)\cap\left( \im\circ f \right)^{-1}\left( \left( c,d \right) \right).
        \end{equation*}

        ($\implies$) Note that
        \begin{equation*}
            \left( \re\circ f \right)^{-1}\left( \left( a,b \right) \right) = f^{-1}\left( V \right)
        \end{equation*}
        where
        \begin{equation*}
            V = \left\lbrace x+iy: a<x<b \right\rbrace.
        \end{equation*}
        Similarly,
        \begin{equation*}
            \left( \im\circ f \right)^{-1}\left( \left( c,d \right) \right) = f^{-1}\left( H \right)
        \end{equation*}
        where
        \begin{equation*}
            H = \left\lbrace x+iy: c<y<d \right\rbrace.
        \end{equation*}
    \end{proof}

    \begin{prop}{}
        Let $\left( X,\tau \right)$ be a topological space. If $f:X\to\F$ is continuous, then $f$ is measurable.
    \end{prop}

    \begin{proof}
        It suffices to check that $f^{-1}\left( U \right)\in\Bor\left( X \right)$ for all open $U\subseteq\F$, which is guaranteed by the continuity of $f$.
    \end{proof}
    
    \begin{prop}{}
        Let $\left( X,\mA \right)$ be a measurable space and let $f,g:X\to\F$ be measurable.
        \begin{enumerate}
            \item For any $\lambda\in\F$, $\lambda f+g$ is measurable.
            \item $fg$ is measurable.
            \item If $g\left( x \right)\neq 0$ for all $x\in X$, then $\frac{1}{g}$ is measurable.
        \end{enumerate}
    \end{prop}
    
    \begin{proof}
        By considering Proposition 2.2, we assume $\F=\R$.

        \begin{enumerate}
            \item Suppose $\lambda>0$. Then given $\alpha\in\R$,
                \begin{equation*}
                    \left( \lambda f \right)^{-1}\left( \left( \alpha,\infty \right) \right) = \left\lbrace x\in X:\lambda f\left( x \right)>\alpha \right\rbrace = \left\lbrace x\in X: f\left( x \right) > \frac{\alpha}{\lambda} \right\rbrace = f^{-1}\left( \left( \frac{\alpha}{\lambda},\infty \right) \right),
                \end{equation*}
                which is measurable.

                In case $\lambda<0$,
                \begin{equation*}
                    \left( \lambda f \right)^{-1}\left( \left( \alpha,\infty \right) \right) = f^{-1}\left( \left( -\infty,\frac{\alpha}{\lambda} \right) \right)
                \end{equation*}
                is measurable.

                When $\lambda = 0$, $\lambda f$ is the constant $0$ function, which is trivially measurable.

                Let $\alpha\in\R$. Then
                \begin{equation*}
                    \begin{aligned}
                        \left( f+g \right)^{-1}\left( \left( \alpha,\infty \right) \right) & = \left\lbrace x\in X:f\left( x \right)+g\left( x \right)>\alpha \right\rbrace = \left\lbrace x\in X: f\left( x \right)>\alpha-g\left( x \right) \right\rbrace \\
                                                                                           & = \bigcup^{}_{q\in\Q} \left( \left\lbrace x\in X: f\left( x \right)>q \right\rbrace\cap \left\lbrace x\in X:g\left( x \right)>\alpha-q \right\rbrace \right) = \bigcup^{}_{q\in\Q} \left( f^{-1}\left( \left( q,\infty \right) \right)\cap g^{-1}\left( \alpha-q,\infty \right) \right),
                    \end{aligned} 
                \end{equation*}
                which is measurable as a countable union of measurable sets.

            \item Note
                \begin{equation*}
                    \left( f+g \right)^{2} = f^{2}+2fg+g^{2}.
                \end{equation*}
                Hence it suffices to show that $f^{2}$ is measurable. Let $\alpha\in\R$.

                Suppose $\alpha\geq 0$. Then
                \begin{equation*}
                    \begin{aligned}
                        f^{-1}\left( \left( \alpha,\infty \right) \right) & = \left\lbrace x\in X: f\left( x \right)^{2} > \alpha \right\rbrace = \left\lbrace x\in X: f\left( x \right) > \sqrt{\alpha} \right\rbrace \cup \left\lbrace x\in X: f\left( x \right) < -\sqrt{\alpha} \right\rbrace \\
                                                                          & = f^{-1}\left( \left( \sqrt{\alpha},\infty \right) \right) \cup f^{-1}\left( \left( -\infty,-\sqrt{\alpha} \right) \right)
                    \end{aligned} 
                \end{equation*}
                is a union of measurable of measurable sets.

                If $\alpha<\infty$, then
                \begin{equation*}
                    \left( f^{2} \right)^{-1}\left( \left( \alpha,\infty \right) \right) = \left\lbrace x\in X: f\left( x \right)^{2}>\alpha \right\rbrace = X
                \end{equation*}
                is measurable.

            \item Let $\alpha\in\R$. Suppose $\alpha>0$. Then
                \begin{equation*}
                    \begin{aligned}
                        \left( \frac{1}{g} \right)^{-1}\left( \left( -\infty,\alpha \right) \right) & = \left\lbrace x\in X: \frac{1}{g\left( x \right)}<\alpha \right\rbrace = \left\lbrace x\in X: g\left( x \right)>\frac{1}{\alpha} \right\rbrace \cup \left\lbrace x\in X: g\left( x \right) < 0 \right\rbrace \\
                                                                                                    & = g^{-1}\left( \left( \frac{1}{\alpha},\infty \right) \right) \cup g^{-1}\left( \left( -\infty,0 \right) \right).
                    \end{aligned} 
                \end{equation*}
                The cases where $\alpha<0,\alpha=0$ are similar.
        \end{enumerate}
    \end{proof}
    
    \begin{notation}{$\overline{\R}$}
        We write $\overline{\R}$ to denote
        \begin{equation*}
            \overline{\R} = \left[ -\infty,\infty \right].
        \end{equation*}
    \end{notation}

    \begin{definition}{\textbf{Borel $\sigma$-algebra} of Subsets of $\overline{\R}$}
        We define the \emph{Borel $\sigma$-algebra} of subsets of $\overline{\R}$, denoted as $\Bor\left( \overline{\R} \right)$, by
        \begin{equation*}
            \Bor\left( \overline{\R} \right) = \left\lbrace A\subseteq\overline{\R}:A\cap\R\in\Bor\left( \R \right) \right\rbrace.
        \end{equation*}
    \end{definition}

    \np To show that $\Bor\left( \overline{\R} \right)$ is \textit{really Borel}, we consider the following metric on $\overline{\R}$. Define
    \begin{equation*}
        \begin{aligned}
            d:\overline{\R}^{2}&\to\left[ 0,\infty \right) \\
            \left( x,y \right) & \mapsto \left| \arctan\left( x \right)-\arctan\left( y \right) \right|,
        \end{aligned} 
    \end{equation*}
    where $\arctan\left( -\infty \right)=-\frac{\pi}{2}, \arctan\left( \infty \right)=\frac{\pi}{2}$.

    \begin{exercise}{}
        Show that $\Bor\left( \overline{\R} \right)$ is generated by the open subsets of $\left( \overline{\R},d \right)$.
    \end{exercise}

    \rruleline

    \np $\Bor\left( \overline{\R} \right)$ is (independently) generated by intervals of the form $\left( \alpha,\infty \right], \left[ -\infty,\alpha \right)$.

    \begin{prop}{}
        Let $\left( f_{n} \right)^{\infty}_{\R}$ be a sequence of measurable functions from $X$ to $\R$.
        \begin{enumerate}
            \item $\sup_{n\in\N}f_n$ is measurable.
            \item $\inf_{n\in\N}f_n$ is measurable.
            \item $\limsup_{n\in\N}f_n$ is measurable.
            \item $\liminf_{n\in\N}f_n$ is measurable.
        \end{enumerate}
    \end{prop}

    \begin{proof}
        \begin{enumerate}
            \item Note that, given $\alpha\in\R$,
                \begin{equation*}
                    \left( \sup_{n\in\N}f_n \right)^{-1}\left( \left( \alpha,\infty \right] \right) = \left\lbrace x\in X: \sup_{n\in\N}f_n\left( x \right)>\alpha \right\rbrace = \bigcup^{}_{n\in\N} \left\lbrace x\in X: f_n\left( x \right)>\alpha \right\rbrace = \bigcup^{}_{n\in\N} f_n^{-1}\left( \left( \alpha,\infty \right) \right).
                \end{equation*}

            \item It suffices to note that $\inf_{n\in\N}f_n = -\sup_{n\in\N}\left( -f_n \right)$.

            \item Recall that
                \begin{equation*}
                    \limsup_{n\in\N} f_n = \lim_{n\to\infty} \sup_{k\geq n} f_k = \inf_{n\in\N}\sup_{k\geq n} f_k.
                \end{equation*}
                Hence by (a), (b), $\limsup_{n\in\N}f_n$ is measurable.

            \item Similar to (c),
                \begin{equation*}
                    \liminf_{n\in\N}f_n = \sup_{n\in\N}\inf_{k\geq n}f_k.
                \end{equation*}
                Hence $\liminf_{n\in\N}f_n$ is measurable.
        \end{enumerate}
    \end{proof}

    \begin{cor}{}
        Let $\left( f_{n} \right)^{\infty}_{n=1}$ be a sequence of measurable functions from $X$ to $\R$. If $f_n\to x$ pointwise, then $f$ is measurable.
    \end{cor}	

    \begin{proof}
        Note that
        \begin{equation*}
            f_n\to x \iff \liminf_{n\in\N}f_n = \limsup_{n\in\N}f_n = \lim_{n\to\infty} f_n.
        \end{equation*}
    \end{proof}
    
    \np Let $\left( X,\mA \right)$ be a measurable space. Then given measurable $f:X\to\F$ and continuous $g:\F\to\F$, $g\circ f$ is measurable, as for any open $U\subseteq\F$,
    \begin{equation*}
        \left( g\circ f \right)^{-1}\left( U \right) = f^{-1}\left( g^{-1}\left( U \right) \right),
    \end{equation*}
    where $g^{-1}\left( U \right)$ is open.

    In particular, this gives alternative proofs that $f^{2}, \frac{1}{f}, \re\circ f, \im\circ f$ are measurable. Moreover, $\left| f \right|$ is measurable.
    
    \begin{definition}{\textbf{$\mu$-almost Everywhere} Predicate}
        Let $\left( X,\mA,\mu \right)$ be a measure space and let $P$ be a predicate on $X$. We say $P$ is true \emph{$\mu$-almost everywhere} (or \emph{$\mu$-ae}) if there exists $N\in\mA$ with $\mu\left( N \right) = 0$ such that $P\left( x \right)$ is true for all $x\in X\setminus N$.
    \end{definition}
    
    \np Note that the definition of $\mu$-almost everywhere does not say that
    \begin{equation*}
        N = \left\lbrace x\in X: \text{$P\left( x \right)$ is false} \right\rbrace
    \end{equation*}
    is measurable. But in case $\mu$ is complete, $N$ is measurable with $\mu\left( N \right) = 0$.

    \begin{prop}{}
        Let $\left( X,\mA,\mu \right)$ be a complete measure space and let $f:X\to\F$ be measurable. Suppose that $g:X\to\F$ is such that $f=g$ $\mu$-ae. Then $g$ is measurable.
    \end{prop}

    \begin{proof}
        Let $N\in\mA$ be such that $\mu\left( N \right) = 0$ with $f=g$ on $X\setminus N$. Then given any measurable $U\subseteq\R$,
        \begin{equation*}
            g^{-1}\left( U \right) = \left( g^{-1}\left( U \right)\cap N \right) \cupdot \left( g^{-1}\left( U \right)\setminus N \right).
        \end{equation*}
        Note that $g^{-1}\left( U \right)\cap N\subseteq N$ so has measure $0$, which means $g^{-1}\left( U \right)\cap N\in\mA$ by the completeness of $\mu$. Moreover, $f=g$ on $X\setminus N$ so that $g^{-1}\left( U \right)\setminus N = f^{-1}\left( U \right)\setminus N$, which is measurable. Thus $g^{-1}\left( U \right)$ is measurable, as required.
    \end{proof}

    \subsection{Simple Approximation}
    
    \begin{definition}{\textbf{Characteristic Function} of a Subset}
        Let $X$ be a set and let $A\subseteq X$. The \emph{characteristic function} of $A$, denoted as $\chi_A$, is defined as
        \begin{equation*}
            \begin{aligned}
                \chi_A:X&\to \R \\
                x&\mapsto
                \begin{cases} 1 & \text{if $x\in A$} \\ 0 & \text{if $x\notin A$} \end{cases}
            \end{aligned} .
        \end{equation*}
    \end{definition}
    
    \np Note that, given $A\subseteq X$,
    \begin{equation*}
        \text{$\chi_A$ is measurable} \iff \text{$A$ is measurable}.
    \end{equation*}
    
    \begin{definition}{\textbf{Simple} Function}
        Let $\left( X,\mA \right)$ be a measurable space. We say $\phi:X\to\F$ is \emph{simple} if
        \begin{equation*}
            \phi = \sum^{n}_{k=1} a_k\chi_{A_k}
        \end{equation*}
        where $a_1,\ldots,a_n\in\F$ and $A_1,\ldots,A_n\in\mA$ are pairwise disjoint.
    \end{definition}
    
    \np Let $\left( X,\mA \right)$ be a measurable space and let $\phi:X\to\F$. Then
    \begin{equation*}
        \text{$\phi$ is simple} \iff \text{$\phi$ is measurable and $\phi\left( X \right)$ is finite}.
    \end{equation*}
    To see the reverse direction, suppose $\phi$ is measurable and $\phi\left( X \right)$ is finite, say
    \begin{equation*}
        \phi\left( X \right) = \left\lbrace a_k \right\rbrace^{n}_{k=1}.
    \end{equation*}
    Then each $A_k = \phi^{-1}\left( \left\lbrace a_k \right\rbrace \right)$ is measurable and $\phi = \sum^{n}_{k=1} a_k\chi_{a_k}$.
    
    \np The goal of this subsection is to show
    \begin{equation*}
        \text{$f:X\to\R$ is measurable} \iff \text{$f$ is a pointwise limit of simple functions}.
    \end{equation*}
    
    \begin{prop}{}
        Let $\left( X,\mA \right)$ be a measurable space and let $f:X\to\R$ be measurable and bounded. Then for all $\epsilon>0$, there are simple $\phi_{\epsilon},\psi_{\epsilon}:X\to\R$ such that
        \begin{enumerate}
            \item $\phi_{\epsilon}\leq f\leq\psi_{\epsilon}$; and
            \item $0\leq\psi_{\epsilon}-\phi_{\epsilon}<\epsilon$.
        \end{enumerate}
    \end{prop}
    
    \begin{proof}
        Let $\epsilon>0$. Say $f\left( X \right)\subseteq\left[ a,b \right)$. Let $y_0,\ldots,y_n$ be given such that
        \begin{equation*}
            a = y_0 < y_1 < \cdots < y_n = b,
        \end{equation*}
        where each $y_k-y_{k-1}<\epsilon$. Let $I_k = \left[ y_{k-1},y_k \right)$. Then each $A_k = f^{-1}\left( I_k \right)$ is measurable. Define
        \begin{equation*}
            \phi = \sum^{n}_{k=1} y_{k-1}\chi_{A_k}, \psi = \sum^{n}_{k=1}y_k\chi_{A_k}.
        \end{equation*}
        Then for any $x\in X$, we have $x\in I_k$ for some $k$, so that $\phi\left( x \right) = y_{k-1} \leq f\left( x \right) \leq y_k = \psi\left( x \right)$.

        Moreover,
        \begin{equation*}
            0 < \psi\left( x \right) - \phi\left( x \right) = y_k-y_{k-1} < \epsilon.
        \end{equation*}
    \end{proof}

    \begin{theorem}{Simple Approximation}
        Let $\left( X,\mA \right)$ be a measure space and let $f:X\to\R$. Then
        \begin{equation*}
            \text{$f$ is measurable} \iff \text{there are simple $\phi_1,\phi_2,\ldots:X\to\R$ with $\phi_n\to f$ pointwise and $\left| \phi_n \right|\leq f$ for all $n\in\N$}.
        \end{equation*}
    \end{theorem}
    
    \begin{proof}
        ($\impliedby$) Recall that pointwise limit of measurable functions is measurable, where each $\phi_n$ is measurable.

        ($\implies$) We split into few cases.

        \begin{case}
            \textit{Suppose $f\geq 0$.}

            Let
            \begin{equation*}
                A_n = \left\lbrace x\in X: f\left( x \right)\leq n \right\rbrace.
            \end{equation*}
            Note that
            \begin{equation*}
                \mA' = \left\lbrace B\cap A_n: B\in\mA \right\rbrace
            \end{equation*}
            is a $\sigma$-algebra of subsets of $A_n$. Then $\left( A_n,\mA' \right)$ is a measurable space and $f|_{A_n}$ is measurable, since
            \begin{equation*}
                \left( f|_{A_n} \right)^{-1}\left( U \right) = f^{-1}\left( U \right)\cap A_n\in\mA'
            \end{equation*}
            for all measurable $U\subseteq\R$. Moreover, by definition $f|_{A_n}$ is bounded.

            Hence by Proposition 2.7, we can find simple $\phi_m,\psi_m:A_n\to\R$, $m\in\N$, such that
            \begin{equation*}
                0\leq\phi_m\leq f\leq\psi_m
            \end{equation*}
            and
            \begin{equation*}
                0\leq\psi_m-\phi_m<\frac{1}{m}
            \end{equation*}
            for all $m\in\N$ on $A_n$.

            Extend $\phi_m\left( x \right) = n$ for all $x\in X\setminus A_n$, so that $\phi_m\leq f$ on $X$. 

            Now fix $x\in X$. Then $x\in A_N$ for some $N$, and so
            \begin{equation*}
                0 \leq f\left( x \right)-\phi_N\left( x \right)\leq\psi_N\left( x \right)-\phi_N\left( x \right)<\frac{1}{N}.
            \end{equation*}
            This means given any $\epsilon>0$ we can take $N'>N$ so that $\frac{1}{N'} < \epsilon$, which means for all $m\geq N'$,
            \begin{equation*}
                0 \leq f\left( x \right) - \phi_m\left( x \right) < \frac{1}{N'} < \epsilon.
            \end{equation*}
            Thus $\phi_m\to f$ pointwise.
        \end{case}

        \begin{case}
            \textit{Consider the general case on $f$. That is, we only assume that $f$ is measurable.}

            Let
            \begin{equation*}
                \begin{aligned}
                    A & = \left\lbrace x\in X: f\left( x \right)\geq 0 \right\rbrace\in\mA \\
                    B & = \left\lbrace x\in X: f\left( x \right)< 0 \right\rbrace\in\mA \\
                \end{aligned} 
            \end{equation*}
            and let $g=f\chi_A, h=-f\chi_B$, so that both $g,h\geq 0$. By Case 1, there exist $\left( \phi_{n} \right)^{\infty}_{n=1}, \left( \psi_{n} \right)^{\infty}_{n=1}$ such that $\phi_n\nearrow g$ and $\psi_n\nearrow h$ pointwise as $n\to\infty$. Then $f=g-h$ so that $\phi_n-\psi_n\to g-h = f$ pointwise. Moreover,
            \begin{equation*}
                \left| \phi_n-\psi_n \right| \leq \left| \phi_n \right| + \left| \psi_n \right| = \phi_n+\psi_n \leq g+h = \left| f \right|.
            \end{equation*}
        \end{case}
    \end{proof}

    \clearpage
    
    \np Note that in the proof, we know that, given a fixed $n\in\N$, we have
    \begin{equation*}
        0\leq f-\phi_m \leq \frac{1}{m}
    \end{equation*}
    on $A_n$. That is,
    \begin{equation*}
        0\leq f\left( x \right)-\phi_m\left( x \right) \leq \frac{1}{m},\hspace{1cm}\forall x\in A_n,
    \end{equation*}
    so that $\phi_m\to f$ \textit{uniformly} as $m\to\infty$ on $A_n$.

    \np Suppose that $f\geq 0$ is measurable and that
    \begin{equation*}
        0\leq \phi_n \leq f,\hspace{1cm}\forall n\in\N
    \end{equation*}
    with $\phi_n\to f$ pointwise. Then by taking $\psi_n = \max\left\lbrace \phi_1,\ldots,\phi_n \right\rbrace$, $\phi_n$ is still simple. Then
    \begin{equation*}
        0\leq\psi_n\leq f,\hspace{1cm}\forall n\in\N
    \end{equation*}
    as well, so that $\psi_n\nearrow f$ pointwise as $n\to\infty$.
    
    \subsection{Two Theorems}
    
    We are going to prove two useful theorems in measure theory in this subsection.

    \begin{lemma}{}
        Let $\left( X,\mA,\mu \right)$ be a finite measure space and let $\left( f_{n} \right)^{\infty}_{n=1}\in \left( \R^{X} \right)^{\N}$ be a sequence of measurable functions such that $f_n\to f$ pointwise for some measurable $f:X\to\R$. Then for every $\alpha,\beta>0$, there exist $B\in\mA, N\in\N$ such that
        \begin{equation*}
            \left| f_n\left( x \right)-f\left( x \right) \right| < \alpha,\hspace{1cm}\forall x\in B,n\geq N
        \end{equation*}
        and
        \begin{equation*}
            \mu\left( X\setminus B \right) < \beta.
        \end{equation*}
    \end{lemma}
    
    \begin{proof}[Proof Sketch]
        Let
        \begin{equation*}
            A_n = \left\lbrace x\in X: \forall k\geq n\left[ f_k\left( x \right)-f\left( x \right)<\alpha \right] \right\rbrace ,\hspace{1cm}\forall n\in\N.
        \end{equation*}
        Then
        \begin{equation*}
            A_n = \bigcap^{}_{k\geq n} \left| f_k-f \right|^{-1}\left( \left( -\infty,\alpha \right) \right),
        \end{equation*}
        which is measurable. Since $f_n\to f$ pointwise, we have
        \begin{equation*}
            X = \bigcup^{\infty}_{n=1} A_n.
        \end{equation*}
        We also have an increasing chain
        \begin{equation*}
            A_1\subseteq A_2\subseteq\cdots,
        \end{equation*}
        so that
        \begin{equation*}
            \lim_{n\to\infty}\mu\left( A_n \right) = \mu\left( \bigcup^{\infty}_{n=1}A_n \right) = \mu\left( X \right) < \infty
        \end{equation*}
        by the continuity from below. Hence we may find $N\in\N$ such that
        \begin{equation*}
            \mu\left( X \right)-\mu\left( A_n \right) < \beta,\hspace{1cm}\forall n\geq N.
        \end{equation*}
        Since $\mu\left( X \right)<\infty$, each $\mu\left( A_n \right)<\infty$ as well, so that
        \begin{equation*}
            \mu\left( X\setminus A_n \right)<\beta,\hspace{1cm}\forall n\geq N.
        \end{equation*}
        By taking $B=A_N$, we are done.
    \end{proof}
    
    \clearpage

    \begin{theorem}{Egoroff}
        Let $\left( X,\mA,\mu \right)$ be a finite measure space and let $\left( f_{n} \right)^{\infty}_{n=1}\in\left( \R^{X} \right)^{\N}$ be a sequence of measurable functions such that $f_n\to f$ pointwise for some measurable $f:X\to\R$. Then for all $\epsilon>0$ there exists $A\in\mA$ such that
        \begin{enumerate}
            \item $f_n\to f$ uniformly on $A$; and
            \item $\mu\left( X\setminus A \right)<\epsilon$.
        \end{enumerate}
    \end{theorem}
    
    \begin{proof}
        Let $\epsilon>0$ be given. For all $n\in\N$, we may find $A_n\in\mA$ and $N_n\in\N$ such that
        \begin{equation*}
            \forall x\in A_n, k\geq N_n\left[ \left| f_k\left( x \right)-f\left( x \right) \right|<\frac{1}{n} \right]
        \end{equation*}
        and
        \begin{equation*}
            \mu\left( X\setminus A_n \right) < \frac{\epsilon}{2^n}.
        \end{equation*}
        Let
        \begin{equation*}
            A = \bigcap^{\infty}_{n=1} A_n.
        \end{equation*}
        Given any $\epsilon'>0$, by taking $n\in\N$ such that $\frac{1}{n}<\epsilon'$, we have, for all $k\geq N_n$ and $x\in A$,
        \begin{equation*}
            \left| f_k\left( x \right)-f\left( x \right) \right|<\frac{1}{n}<\epsilon'.
        \end{equation*}
        Hence $f_k\to f$ uniformly on $A$. Finally,
        \begin{equation*}
            \mu\left( X\setminus A \right) = \mu\left( \bigcup^{\infty}_{n=1}\left( X\setminus A_n \right) \right) \leq \sum^{\infty}_{n=1}\mu\left( X\setminus A_n \right) < \sum^{\infty}_{n=1} \frac{\epsilon}{2^n} = \epsilon.
        \end{equation*}
    \end{proof}
    
    \np Let $m$ be the Lebesgue measure on $\R$ and let $A\subseteq\R$ with $m\left( A \right)<\infty$. Let $\left( f_{n} \right)^{\infty}_{n=1}$ be a sequence of measurable functions from $A$ to $\R$ that converges to $f:A\to\R$. Then by Egoroff's theorem, for every $\epsilon>0$, there is $B\subseteq A$ such that
    \begin{equation*}
        f_n\to f\text{ uniformly on }B
    \end{equation*}
    and
    \begin{equation*}
        m\left( A\setminus B \right) < \frac{\epsilon}{2}.
    \end{equation*}
    Then we can find a closed subset $C\subseteq B$ with
    \begin{equation*}
        m\left( B\setminus C \right) < \frac{\epsilon}{2}
    \end{equation*}
    by the regularity of Lebesgue measure. Then
    \begin{equation*}
        f_n\to f\text{ uniformly on }C
    \end{equation*}
    and
    \begin{equation*}
        m\left( A\setminus C \right) = m\left( A\setminus B \right) + m\left( B\setminus C \right) < \epsilon.
    \end{equation*}
    Hence for the Lebesgue measure (in fact, any Lebesgue-Steltjes measure), we can assume that $f_n\to f$ uniformly on a closed set with arbitrarily small difference.
    
    \begin{lemma}{}
        Let $A\subseteq\R$ be Lebesgue measurable and let $\phi:A\to\R$ be Lebesgue-simple. Then for all $\epsilon>0$, there exists closed $C\subseteq\R$ and a continuous $g:\R\to\R$ such that
        \begin{enumerate}
            \item $C\subseteq A$;
            \item $\phi=g$ on $C$; and
            \item $m\left( A\setminus C \right)<\epsilon$.
        \end{enumerate}
    \end{lemma}

    \clearpage

    \begin{proof}
        Write
        \begin{equation*}
            \phi = \sum^{n}_{i=1} a_i\chi_{A_i},
        \end{equation*}
        where each $a_i\neq 0$ and $A_i = \phi^{-1}\left( \left\lbrace a_i \right\rbrace \right)$. Let $A_0 = \phi^{-1}\left( \left\lbrace 0 \right\rbrace \right)$. We also insist that $a_i\neq a_j$ for $i\neq j$. Then
        \begin{equation*}
            A = \bigcupdot^{n}_{i=0} A_i.
        \end{equation*}
        Let $\epsilon>0$ be given. For each $i$, let $C_i$ be a closed such that $C_i\subseteq A_i$ and
        \begin{equation*}
            m\left( A_i\setminus C_i \right) < \frac{\epsilon}{n+1}
        \end{equation*}
        by regularity of Lebesgue measure. Let
        \begin{equation*}
            C = \bigcupdot^{n}_{i=0}C_i,
        \end{equation*}
        which is closed. Since $\phi$ is continuous on each $C_i$ and $C_i\cap C_j=\emptyset$, $\phi$ is continuous on $C$. Then there is continuous $g:\R\to\R$ that extends $\phi:C\to\R$. Finally,
        \begin{equation*}
            m\left( A\setminus C \right) = m\left( \bigcupdot^{n}_{i=0}A_i\setminus C_i \right) = \sum^{n}_{i=0} m\left( A_i\setminus C_i \right) < \epsilon.
        \end{equation*}
    \end{proof}
    
    \begin{theorem}{Lusin}
        Let $f:A\to\R$ be Lebesgue measurable. Then for all $\epsilon>0$, there exists continuous $g:\R\to\R$ and closed $C\subseteq\R$ such that
        \begin{enumerate}
            \item $C\subseteq A$;
            \item $f=g$ on $C$; and
            \item $m\left( A\setminus C \right)<\epsilon$.
        \end{enumerate}
    \end{theorem}
    
    \begin{proof}
        We split the proof into two cases. Let $\epsilon>0$ be given.

        \begin{case}
            \textit{Suppose $m\left( A \right)<\infty$.}

            Let $\left( \phi_{n} \right)^{\infty}_{n=1}$ be a sequence of simple functions such that $\phi_n\to f$ pointwise by simple approximation. For each $n\in\N$, let $C_n\subseteq\R$ be closed and $g_n:\R\to\R$ be continuous such that $\phi_n=g_n$ on $C_n$ and
            \begin{equation*}
                m\left( A\setminus C_n \right) < \frac{\epsilon}{2^{n+1}}.
            \end{equation*}
            By Egoroff, let $C_0$ be the closed set such that 
            \begin{equation*}
                \text{$\phi_n\to f$ uniformly on $C_0$}
            \end{equation*}
            and
            \begin{equation*}
                m\left( A\setminus C_0 \right) < \frac{\epsilon}{2}.
            \end{equation*}
            Let
            \begin{equation*}
                C = \bigcap^{\infty}_{n=0} C_n.
            \end{equation*}
            Then,
            \begin{equation*}
                g_n = \phi_n \to f\text{ uniformly on }C.
            \end{equation*}
            In particular, $f$ is continous on $C$. This means we can extend $f|_C$ to continuous $g:\R\to\R$. Finally,
            \begin{equation*}
                m\left( A\setminus C \right) = m\left( A\setminus\bigcap^{\infty}_{n=0}C_n \right) = m\left( \bigcup^{\infty}_{n=0}\left( A\setminus C_n \right) \right) \leq m\left( A\setminus C_0 \right) + \sum^{\infty}_{n=1}m\left( A\setminus C_n \right) < \epsilon.
            \end{equation*}
        \end{case}

        \begin{case}
            \textit{Suppose $m\left( A \right)<\infty$.}

            This is left as an exercise.
        \end{case}
    \end{proof}
    
    
    
    
    
    
    
    
    
    
    
    
    
    
    
    
    
    
    
    
    
    
    
    
    
    
    
    
    
    
    

\end{document}
