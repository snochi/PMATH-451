\documentclass[pmath451]{subfiles}

%% ========================================================
%% document

\begin{document}

    \section{Product Measures}
    
    \subsection{Product Measures}

    \begin{definition}{\textbf{Measurable Rectangle}}
        Let $\left( X,\mA,\mu \right),\left( Y,\mB,\nu \right)$ be measure spaces. For every $A\in\mA,B\in\mB$, we call $A\times B$ a \emph{measurable rectangle}.
    \end{definition}
    
    \begin{lemma}{}
        Let $\left( X,\mA,\mu \right),\left( Y,\mB,\nu \right)$ be measure spaces and let $\left\lbrace A_k\times B_k \right\rbrace^{\infty}_{k=1}$ be a collection of measurable rectangles that are pairwise disjoint. Also assume that
        \begin{equation*}
            \bigcupdot^{\infty}_{k=1} A_k\times B_k = A\times B
        \end{equation*}
        for some $A\in\mA,B\in\mB$. Then
        \begin{equation*}
            \mu\left( A \right)\nu\left( B \right) = \sum^{\infty}_{k=1} \mu\left( A_k \right)\nu\left( B_k \right).\footnotemark[1]
        \end{equation*}
        
        \noindent
        \begin{minipage}{\textwidth}
            \footnotetext[1]{We are using the convention $0\infty = 0$.}
        \end{minipage}
    \end{lemma}
    
    \begin{proof}
        Fix $x\in A$. For all $y\in B$, there exists a unique $k\in\N$ such that $\left( x,y \right)\in A_k\times B_k$. Hence
        \begin{equation*}
            B = \bigcupdot^{}_{k\in\N:x\in A_k} B_k
        \end{equation*}
        This means
        \begin{equation*}
            \mu\left( B \right) = \sum^{}_{k\in\N:x\in A_k} \mu\left( B_k \right),
        \end{equation*}
        so that
        \begin{equation*}
            \nu\left( B \right)\chi_A\left( x \right) = \sum^{\infty}_{k=1}\nu\left( B_k \right)\chi_{A_k},\hspace{1cm}\forall x\in X.
        \end{equation*}
        By MCT,
        \begin{equation*}
            \nu\left( B \right)\mu\left( A \right) - \sum^{\infty}_{k=1} \sum^{\infty}_{k=1}\nu\left( B_k \right)\mu\left( A_k \right).
        \end{equation*}
    \end{proof}
    
    \np Let
    \begin{equation*}
        \mR = \left\lbrace \bigcupdot^{n}_{k=1} A_k\times B_k : n\geq 0 , \forall k\in\left\lbrace 1,\ldots,n \right\rbrace\left[ A_k\in\mA,B_k\in\mB \right] \right\rbrace.
    \end{equation*}
    
    \begin{prop}{}
        Let
        \begin{equation*}
            \begin{aligned}
                \lambda:\mR&\to\left[ 0,\infty \right] \\
                \bigcupdot^{n}_{k=1} A_k\times B_k&\mapsto \sum^{n}_{k=1}\mu\left( A_k \right)\nu\left( B_k \right)
            \end{aligned} .
        \end{equation*}
        Then $\lambda$ is a premeasure.
    \end{prop}

    \rruleline

    \np By Caratheodory, there is a complete measure
    \begin{equation*}
        \left( X\times Y, \overline{\mA\times\mB}, \mu\times\nu \right)
    \end{equation*}
    on $X\times Y$ such that
    \begin{equation*}
        \mA\times\mB \subseteq \overline{\mA\times\mB} = \left\lbrace A\times B\in\mA\times\mB : A\times B\text{ is $\lambda^{*}$-measurable} \right\rbrace.
    \end{equation*}
    and
    \begin{equation*}
        \left( \mu\times\nu \right)\left( A\times B \right) = \mu\left( A \right)\nu\left( B \right), \hspace{1cm}\forall A\in\mA, B\in\mB.
    \end{equation*}
    
    \begin{definition}{\textbf{Product Measure}}
        Consider the above setting. We call $\mu\times\nu$ the \emph{product measure} on $\mA\times\mB$.
    \end{definition}
    
    \subsection{Product Integration}
    
    \begin{theorem}{Fubini}
        Let $\left( X,\mA,\mu \right),\left( Y,\mB,\nu \right)$ be complete measure spaces. If $f\in\Lone\left( X\times Y,\overline{\mA\times\mB},\mu\times\nu \right)$, then
        \begin{enumerate}
            \item For all $x\in X$, let
                \begin{equation*}
                    \begin{aligned}
                        f_x:Y&\to\F \\
                        y&\mapsto f\left( x,y \right)
                    \end{aligned}.
                \end{equation*}
                Then $f_x\in\Lone\left( Y,\mB,\nu \right)$ for almost all $x$.
            \item For all $y\in Y$, let
                \begin{equation*}
                    \begin{aligned}
                        f^y:X&\to\F \\
                        x&\mapsto f\left( x,y \right)
                    \end{aligned}.
                \end{equation*}
                Then $f^y\in\Lone\left( X,\mA,\mu \right)$ for almost all $y$.
            \item Let
                \begin{equation*}
                    \begin{aligned}
                        F:X&\to\F \\
                        x&\mapsto\int f_xd\nu
                    \end{aligned} .
                \end{equation*}
                Then $F\in\Lone\left( X,\mA,\nu \right)$.
            \item Let
                \begin{equation*}
                    \begin{aligned}
                        G:X&\to\F \\
                        y&\mapsto\int f^yd\mu
                    \end{aligned} .
                \end{equation*}
                Then $G\in\Lone\left( X,\mA,\nu \right)$.
            \item We have
                \begin{equation*}
                    \int_{X\times Y} fd\left( \mu\times\nu \right) = \int_X\int_Y f\left( x,y \right)d\nu d\mu = \int_Y\int_Xf\left( x,y \right)d\mu d\nu.
                \end{equation*}
        \end{enumerate}
    \end{theorem}

    \rruleline
    
    \np Given $E\subseteq X\times Y$, let us write write
    \begin{equation*}
        E_x = \left\lbrace y\in Y: \left( x,y \right)\in E \right\rbrace, \hspace{1cm}\forall x\in X
    \end{equation*}
    and
    \begin{equation*}
        E^y = \left\lbrace x\in X: \left( x,y \right)\in E \right\rbrace,\hspace{1cm}\forall y\in Y.
    \end{equation*}

    \begin{lemma}{}
        Let $\left( X,\mA,\mu \right),\left( Y,\mB,\nu \right)$ be measure spaces and let
        \begin{equation*}
            \mR = \left\lbrace \bigcupdot^{n}_{k=1}A_k\times B_k : n\geq 0, \forall k\in\left\lbrace 1,\ldots,n \right\rbrace\left[ A_k\in\mA,B_k\in\mB \right] \right\rbrace.
        \end{equation*}
        Let $E\in\mR_{\sigma\delta}$ with $\left( \mu\times\nu \right)\left( E \right) < \infty$. Then
        \begin{enumerate}
            \item $g:X\to\R$ by $g\left( x \right) = \nu\left( E_x \right)$ for all $x\in X$ is $\mu$-measurable;
            \item $g\in\Lplus\cap\Lone$; and
            \item $\int gd\mu = \left( \mu\times\nu \right)\left( E \right)$.
        \end{enumerate}
    \end{lemma}
    
    \begin{proof}
        
        \begin{case}
            \textit{Suppose $E=A\times B$ for some $A\in\mA,B\in\mB$.}

            Then
            \begin{equation*}
                E_x = \begin{cases} B & \text{if $x\in A$} \\ \emptyset & \text{if $x\notin A$} \end{cases} \in\mB,  \hspace{1cm}\forall x\in X
            \end{equation*}
            Now
            \begin{equation*}
                g\left( x \right) = \nu\left( E_x \right) = \nu\left( B \right)\chi_A\left( x \right),\hspace{1cm}\forall x\in X
            \end{equation*}
            so that $g$ is a nonnegative measurable function, with
            \begin{equation*}
                \int gd\mu = \int\nu\left( B \right)\chi_Ad\mu = \nu\left( B \right)\mu\left( A \right) = \left( \mu\times \nu \right)\left( E \right)<\infty,
            \end{equation*}
            as needed.
        \end{case}

        \begin{case}
            \textit{Consider $E=\bigcup^{\infty}_{i=1}A_i\times B_i$ for some $A_1,\ldots\in\mA, B_1,\ldots\in\mB$.}

            Without loss of generality, we may assume that the union is disjoint, since intersection of rectangles is still a rectangle.

            Define $g_i=\nu\left( B_i \right)\chi_{A_i}$ for all $i\in\N$. Then
            \begin{equation*}
                g = \sum^{\infty}_{i=1} g_i
            \end{equation*}
            so that $g$ is $\mu$-measurable. Moreover, every $E_x = \bigcupdot^{\infty}_{i=1} \left( A_i\times B_i \right)_x$ is measurable.

            Then by the MCT,
            \begin{equation*}
                \int gd\mu = \sum^{\infty}_{i=1} \int g_id\mu = \sum^{\infty}_{i=1}\mu\left( A_i \right)\nu\left( B_i \right) = \sum^{\infty}_{i=1}\left( \mu\times\nu \right)\left( A_i\times B_i \right) = \left( \mu\times\nu \right)\left( E \right) < \infty.
            \end{equation*}
        \end{case}

        \begin{case}
            \textit{Consider $E=\bigcap^{\infty}_{n=1}E_n$, where each $E_n\in\mR_{\sigma}$.}

            Without loss of generality, we may assume
            \begin{equation*}
                E_1\supseteq E_2\supseteq\cdots.
            \end{equation*}
            Moreover, we may also assume that
            \begin{equation*}
                \left( \mu\times\nu \right)\left( E_1 \right) < \infty,
            \end{equation*}
            since $\left( \mu\times\nu \right)\left( E \right)<\infty$.

            Then we have that
            \begin{equation*}
                E_x = \bigcap^{\infty}_{n=1} \left( E_n \right)_x
            \end{equation*}
            and
            \begin{equation*}
                \left( E_1 \right)_x\supseteq\left( E_2 \right)_x\supseteq\cdots,
            \end{equation*}
            so
            \begin{equation*}
                \lim_{n\to\infty}\nu\left( \left( E_n \right)_x \right) = \nu\left( E \right)
            \end{equation*}
            and
            \begin{equation*}
                \lim_{n\to\infty}\left( \mu\times\nu \right)\left( E_n \right) = \left( \mu\times\nu \right)\left( E \right).
            \end{equation*}

            Let
            \begin{equation*}
                \begin{aligned}
                    g_n:X&\to\R\\
                    x&\mapsto \nu\left( \left( E_n \right)_x \right)
                \end{aligned} ,\hspace{1cm}\forall n\in\N.
            \end{equation*}
            Then $0\geq g$ and $g_n\searrow g$ pointwise with
            \begin{equation*}
                \int g_1d\nu = \left( \mu\times\nu \right)\left( E_1 \right) < \infty,
            \end{equation*}
            so by the LDCT, 
            \begin{equation*}
                \int gd\mu = \lim_{n\to\infty}\int g_nd\mu = \lim_{n\to\infty}\left( \mu\times\nu \right)\left( E_n \right) = \left( \mu\times\nu \right)\left( E \right).
            \end{equation*}
        \end{case}
    \end{proof}



    
    
    
    
    
    
    
    
    
    
    
    
    
    
    
    
    
    
    
    
    
    
    
    
    
    
    

\end{document}
